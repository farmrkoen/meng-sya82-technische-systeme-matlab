\section{Einleitung}

\subsection{Relevanz und Problemstellung}
Seit seiner Erfindung im Jahre 1976 durch Nikolaus August Otto hat der Viertakt-Ottomotor als Antriebsaggregat eine weitreichende Verwendung gefunden.\footcite[Vgl.][1]{merker2019} 
Obwohl die Elektromobilität an Relevanz gewinnt setzt sich diese Bedeutung auch heute weiter fort.
Daten des Kraftfahrt-Bundesamtes zeigen, dass Benzinmotoren bei den Neuzulassungen in Deutschland weiterhin einen erheblichen Anteil ausmachen.\footcite[Vgl.][]{kba2025}
Die hohe Energiedichte sowie einfache Möglichkeiten zum Transport und Speichern machen Flüssigbrennstoff zur bevorzugten Energiequelle für den Verkehr.\footcite[Vgl.][2]{leach2020}
Doch die Anforderungen haben sich angesichts des Klimawandels und schlechter Luftqualität durch $\text{CO}_2$-Emissionen gewandelt.
Eine hohe Last und Motordrehzahl erhöhen die Emission von $\text{CO}_2$\footcite[Vgl.][6]{shahad2015}, weshalb eine präzise Regelung der Motordrehzahl unter variierender Last einen großen Stellenwert einnimmt.

Die experimentelle Entwicklung von Regelungsstrategien am physischen Prüfstand erweisen sich als aufwändiger Prozess.
An dieser Stelle ist die Modellbildung und Simulation ein unverzichtbares Werkzeug.
Dynamische Systemmodelle ermöglichen die virtuelle Nachbildung und Analyse des komplexen Zusammenspiels von Thermodynamik, Mechanik und Regelungstechnik.

Aus diesem Grund widmet sich diese Arbeit der Erweiterung eines bestehenden thermodynamischen Modells eines Viertakt-Ottomotors zu einem dynamischen System.
Ziel wird es sein, das Verhalten des Motors bei einem abrupten Lastwechesel zu simulieren.
Darauf aufbauend soll mittels PID-Reglers eine Drehzahlstabilisierung implementiert werden.
Damit soll demonstriert werden, wie der Motor stets im optimalen Betriebspunkt gehalten wird, um die Energieeffizienz zu steigern.

\subsection{Vorgehensweise und Aufbau der Arbeit}

